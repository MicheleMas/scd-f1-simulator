\chapter{Introduzione} % Main chapter title

\label{Chapter1} % Change X to a consecutive number; for referencing this chapter elsewhere, use \ref{ChapterX}

\lhead{Capitolo 1. \emph{Introduzione}} % Change X to a consecutive number; this is for the header on each page - perhaps a shortened title

Questo documento contiene la relazione finale del progetto di Sistemi Concorrenti e Distribuiti, riguardante la progettazione e realizzazione di un simulatore concorrente e distribuito di una competizione sportiva assimilabile a quelle automobilistiche di Formula 1.
I requisiti del sistema sono i seguenti:
\begin{itemize}
 \item un circuito, possibilmente selezionabile in fase di configurazione, dotato almeno della pista e della corsia di rifornimento, ciascuna della quali soggette a regole congruenti di accesso, condivisione, tempo di percorrenza, condizioni atmosferiche, ecc.
 \item un insieme configurabile di concorrenti, ciascuno con caratteristiche specifiche di prestazione, risorse, strategia di gara, ecc.
 \item un sistema di controllo capace di riportare costantemente, consistentemente e separatamente, lo stato della competizione, le migliori prestazioni (sul giro, per sezione di circuito) e anche la situazione di ciascun concorrente rispetto a specifici parametri tecnici
 \item una particolare competizione, con specifica configurabile della durata e controllo di terminazione dei concorrenti a fine gara.
\end{itemize}
--TODO inserire che vogliamo che ci possano essere più di un monitor e i controller per giocare col Joystick. SCIMMIE!
